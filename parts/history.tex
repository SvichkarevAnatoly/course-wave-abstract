\chapter*{Теория возмущений}
\addcontentsline{toc}{section}{Теория возмущений}

Теория возмущений возникла при решение
проблем небесной механики,
которая изучает влияние планет
на движение вокруг Солнца одной исследуемой планеты.
Так возникла одна из классических задач трёх тел.
Например, при изучении системы Луна-Земля-Солнце
в качестве малого параметра выбиралось
отношение масс Луны и Земли.
В работах Лагранжа и Лапласа было
выдвинуто предположение о том,
что постоянные величины,
характеризующие движение планеты вокруг Солнца,
ввиду влияния движения других планет <<возмущаются>>
и претерпевают изменения, зависящие от времени.
Таким образом сформировалось понятие <<теория возмущений>>.
В результате работ Лагранжа, Лапласа, Пуассона, Гаусса
оказалось возможным проводить вычисления с высокой точностью.
Таким образом удалось открыть планету Нептун
и проанализировать отклонения траектории планеты Уран \cite{vinogradov1977}.

Одним из признаков применимости теории возмущения
является условие того, что уравнения,
описывающие исследуемый процесс,
содержат в явной или неявной форме малый параметр
(или несколько таких параметров).
При этом требуется, чтобы при нулевом значении
малого параметра уравнения допускали точное решение,
и таким образом проблема сводится к нахождению
асимптотики наилучшего приближения к истинному решению
с точностью до $\ep, \ep^2, \dots$.

\section*{Секулярные члены}
\addcontentsline{toc}{subsection}{Секулярные члены}

Трудности первоначально разработанных методов теории возмущений
были обусловлены наличием в получающихся разложениях членов,
содержащих время $t$ вне знака $\sin$ и $\cos$.

Поправочный член будет мал,
только тогда, когда произведение $\ep t$
мало по сравнению с единицей.
В случае, если величина $\ep t$
имеет порядок $O(1)$, то член,
относительно которого предполагается,
что он должен быть малой поправкой,
оказывается того же порядка,
что и главный член разложения.
Если же $\ep t > O(1)$,
то поправка может оказаться
даже больше главного члена разложения.
Поэтому разложение будет применимо только
для таких времён $t$, при которых $\ep t < O(1)$.
Подобные разложения являются неравномерными по $t$,
поскольку при больших временах $t$ их справедливость нарушается,
такие разложения называются \textbf{медленными}.
В астронамической литературе такие члены
принято называть \textbf{вековыми} или
\textbf{секулярными} (от франц. век, столетие).
Название появилось в астронамических приложениях,
где величина $\ep$ оказывается обычно крайне малой,
так что произведение $\ep t$ начинает играть
заметную роль в расчётах лишь по истечении
очень большого промежутка времени, порядка столетия
\cite{nayfeh1984}.

Вклад таких членов в ряд существеннен лишь
за длительные промежутки времени (порядка столетий),
но и в этом случае невозможно строгое описание планетных
траекторий в схеме теории возмущений ---
приемлемым является только первое приближение.
Появление так называемых \textbf{секулярных членов}
обусловлено зависимостью частоты движения (обращения)
исследуемой планеты от соответствующих частот других планет.


Учёт такого рода зависимости и
приводит к возникновению в решениях
как \textbf{секулярных (вида $A t^n$)},
так и смешанных (вида $B t \cos(\omega t + \phi)$) членов.

Например, соотношение:
\begin{equation} \label{Eq:omega}
    \omega = \omega_0 + \ep \omega_1 
\end{equation}

В схеме теории возмущений допускает
следующее разложение по $\ep \, (\ep \ll 1)$:
\begin{equation*}
    \sin(\omega t) = \sin(\omega_0 t) +
    \ep \omega_1 t \cos(\omega_0 t) + \dots,
\end{equation*}
смешанный член в котором появляется в результате разложения
колебания с частотой \eqref{Eq:omega}
по колебаниям с частотой $\omega_0$.

