\chapter*{Теория возмущений}
\addcontentsline{toc}{section}{Теория возмущений}

Теория возмущений возникла при решение
проблем небесной механики,
которая изучает влияние планет
на движение вокруг Солнца одной исследуемой планеты.
Так возникла одна из классических задач трёх тел.
Например, при изучении системы Луна-Земля-Солнце
в качестве малого параметра выбиралось
отношение масс Луны и Земли.
В работах Лагранжа и Лапласа было
выдвинуто предположение о том,
что постоянные величины,
характеризующие движение планеты вокруг Солнца,
ввиду влияния движения других планет <<возмущаются>>
и претерпевают изменения, зависящие от времени.
Таким образом сформировалось понятие <<теория возмущений>>.
В результате работ Лагранжа, Лапласа, Пуассона, Гаусса
оказалось возможным проводить вычисления с высокой точностью.
Таким образом удалось открыть планету Нептун
и проанализировать отклонения траектории планеты Уран \cite{vinogradov1977}.

Трудности первоначально разработанных методов теории возмущений
были обусловлены наличием в получающихся разложениях членов,
содержащих время $t$ вне знака $\sin$ и $\cos$.
Вклад таких членов в ряд существеннен лишь
за длительные промежутки времени (порядка столетий),
но и в этом случае невозможно строгое описание планетных
траекторий в схеме теории возмущений ---
приемлемым является только первое приближение.
Появление так называемых \textbf{секулярных членов}
обусловлено зависимостью частоты движения (обращения)
исследуемой планеты от соответствующих частот других планет.

Учёт такого рода зависимости и
приводит к возникновению в решениях
как \textbf{секулярных (вида $A t^n$)},
так и смешанных (вида $B t \cos(\omega t + \phi)$) членов.

Например, соотношение:
\begin{equation} \label{Eq:omega}
    \omega = \omega_0 + \varepsilon \omega_1 
\end{equation}

В схеме теории возмущений допускает
следующее разложение по $\varepsilon \, (\varepsilon \ll 1)$:
\begin{equation*}
    \sin(\omega t) = \sin(\omega_0 t) +
    \varepsilon \omega_1 t \cos(\omega_0 t) + \dots,
\end{equation*}
смешанный член в котором появляется в результате разложения
колебания с частотой \eqref{Eq:omega}
по колебаниям с частотой $\omega_0$.

Одним из признаков применимости теории возмущения
является условие того, что уравнения,
описывающие исследуемый процесс,
содержат в явной или неявной форме малый параметр
(или несколько таких параметров).
При этом требуется, чтобы при нулевом значении
малого параметра упарнения допускали точное решение,
и таким образом проблема сводится к нахождению
асимптотики наилучшего приближения к истинному решению
с точностью до $\varepsilon, \varepsilon^2, \dots$.
