\chapter*{Введение}
\addcontentsline{toc}{section}{Введение}

Важность асимптотических методов
в теории дифференциальных уравнений
была понята математиками
во второй половине девятнадцатого столетия,
и значительная часть современной асимптотической теории
была создана именно тогда.
В последнее время стало ясно,
насколько важны асимптотические ряды
для понимания структуры решений обыкновенных дифференциальных уравнений
и что они неизбежно возникают во многих вопросах прикладной математики
\cite{vazov1968}.

Многие задачи прикладной математики, физики и других областей
не позволяют получить точные аналитические решения.
Если даже решение найдено,
оно может оказаться малополезным для
математической и физической интерпретации
или численных расчётов.
Таким образом, для получения информации
о решениях уравнений приходится
обращаться к аппроксимациям,
численным решениям или к их сочетанию.

В настоящее время, даже с таким развитием
вычислительной техники,
методы малого параметра не утратили свою силу.
Они используются для выявления качественных особенностей задач.

Среди приближённых методов
стоит отметить асимптотические методы возмущений
(методы асимптотических разложений).
Для качественного и количественного представления
решения методом асимптотического разложения,
даже если они расходятся,
могут оказаться более полезными,
чем равномерно и абсолютно сходящиеся разложения
\cite{nayfeh1977}.

\section*{Формулировка задания}
\addcontentsline{toc}{subsection}{Формулировка задания}

Изучить метод асимптотических возмущений
и метод многомасштабных разложений
для обыкновенных дифференциальных уравнений.
Ознакомиться с понятиями медленных переменных
и секулярных слагаемых.
Получить асимптотическое решение  $O(\ep^2)$
в виде бегущей волны, зависящей от $z=x-Vt$,
для возмущенного уравнения КдВ:
\begin{equation*}
    u_t + u u_x + b u_{xxx} = \ep f(u),
\end{equation*}
где
\begin{equation*}
    f(u) = - a_1 u + a_2 u^2 - a_3 u^3,
\end{equation*}
вводя дополнительные (медленные) переменные
для исключения секулярных слагаемых.

