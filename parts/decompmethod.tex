\chapter*{Метод многомасштабных разложений}
\addcontentsline{toc}{section}{Метод многомасштабных разложений}

Для многих физических задач характерно
наличие малой силы или возмущения,
действующих в течение длительного времени.
Метод многомасштабных разложений разработан
для систематического исследования
такого кумулятивного эффекта.
Цель подобных методов состоит
в построении разложений,
равномерно пригодных на больших интервалах времени.
\cite{coul1972}

Основой метода --- представление
искомой зависимости $u(t)$ как сложной:
\begin{equation*}
    u = u(T_0(t, \lambda), T_1(t, \lambda), \dots),
\end{equation*}

где масштабы $T_0, T_1, \dots$
имеют разный порядок при $\lambda \to 0$.
Часто достаточно принять
$T_0 = t, \, T_1 = \lambda t, \, T_2 = \lambda^2 t, \, \dots$

Оператор дифференцирования при простейшем выборе масштабов
становится следующим:
\begin{equation*}
    \dv{}{t} = \pdv{}{T_0} + \lambda \pdv{}{T_1} +
    \lambda^2 \pdv{}{T_2} + \dots,
\end{equation*}

из обыкновенного дифференциального уравнения
получается уравнение в частных производных.

Рассмотрим пример нелинейных колебаний \cite{eliseev1999}.

Решение уравнения:
\begin{equation*}
    \ddot{u} + u = \lambda f(u, \dot{u})
\end{equation*}



