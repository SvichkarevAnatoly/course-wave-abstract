\chapter*{Метод многомасштабных разложений}
\addcontentsline{toc}{section}{Метод многомасштабных разложений}

Для многих физических задач характерно
наличие малой силы или возмущения,
действующих в течение длительного времени.
Метод многомасштабных разложений разработан
для систематического исследования
такого кумулятивного эффекта.
Цель подобных методов состоит
в построении разложений,
равномерно пригодных на больших интервалах времени.
\cite{coul1972}

Основой метода --- представление
искомой зависимости $u(t)$ как сложной:
\begin{equation*}
    u = u(T_0(t, \ep), T_1(t, \ep), \dots),
\end{equation*}

где масштабы $T_0, T_1, \dots$
имеют разный порядок при $\ep \to 0$.
Часто достаточно принять
$T_0 = t, \, T_1 = \ep t, \, T_2 = \ep^2 t, \, \dots$

Оператор дифференцирования при простейшем выборе масштабов
становится следующим:
\begin{equation*}
    \dv{}{t} = \pdv{}{T_0} + \ep \pdv{}{T_1} +
    \ep^2 \pdv{}{T_2} + \dots,
\end{equation*}

из обыкновенного дифференциального уравнения
получается уравнение в частных производных.

\chapter*{Рассмотрение примера}
\addcontentsline{toc}{subsection}{Рассмотрение примера}

Пример взят из \cite{nayfeh1984}.

Рассмотрим уравнение:
\begin{equation} \label{Eq:example}
    \ddot{u} + u + \ep u^3 = 0
\end{equation}

Выпишем его равномерное разложение:
\begin{equation*}
    u = a \cos(t + \beta + \frac{3}{8} \ep t a^2) +
    \frac{1}{32} \ep a^3
    \cos(3 t + 3 \beta + \frac{9}{8} \ep t a^2) + \dots
\end{equation*}

Видно, что мы не можем разделить
функциональную зависимость $u$ от $t$ и $\ep$,
поскольку функция $u$ зависит не только от этих аргументов
по отдельности, но и от произведения $\ep t$.
Если посмотреть на следующие члены более высокого порядка,
то можно понять, что $u$ зависит и от комбинаций
$\ep t, \, \ep^2 t, \, \ep^3 t, \, \dots$.

Это означает, что:
\begin{equation*}
    u(t, \ep) = u(t, \ep t, \ep^2 t, \ep^3 t, \dots, \ep) 
\end{equation*}

или
\begin{equation*}
    u(t, \ep) = u(T_0, T_1, T_2, T_3, \dots, \ep),
\end{equation*}

где аргументы $T_n$ определяются следующим образом:
\begin{equation*}
    T_0 = t, \quad T_1 = \ep t, \quad T_2 = \ep^2 t,
    \quad T_3 = \ep^3 t, \quad \dots
\end{equation*}

Так как $\ep$ является малым параметром,
то величины $T_n$ представляют собой
\textbf{разные временные масштабы исходной задачи}.

Перейдём в исходном уравнении \eqref{Eq:example}
от независимой переменной $t$ к переменным $T_n$.
Используя правило дифференцирования сложной функции, получаем:
\begin{equation*}
    \begin{gathered}
        \dv{}{t} = \pdv{}{T_0} + \ep \pdv{}{T_1} +
        \ep^2 \pdv{}{T_2} + \dots, \\
        \dv[2]{}{t} = \pdv[2]{}{T_0} +
        2 \ep \pdv{}{T_0}{T_1} +
        \ep^2 \left( 2 \pdv{}{T_0}{T_2} + \pdv[2]{T_1}\right) + \dots. 
    \end{gathered}
\end{equation*}

При этом уравнение \eqref{Eq:example} принимает вид:
\begin{equation} \label{Eq:exPart}
    \pdv[2]{u}{T_0} = 2 \ep \pdv{u}{T_0}{T_1} +
    \ep^2 \left( 2 \pdv{u}{T_0}{T_2} + \pdv[2]{u}{T_1} \right) +
    u + \ep u^3 + \dots = 0.
\end{equation}

Будем искать приближённое решение уравнения \eqref{Eq:exPart} в виде:
\begin{equation} \label{Eq:insertDecomp}
    u = u_0(T_0, T_1, T_2, \dots) +
    \ep u_1(T_0, T_1, T_2, \dots) + \dots.
\end{equation}

Подстановка этого разложения в \eqref{Eq:exPart} даёт:
\begin{equation*}
    \pdv[2]{u_0}{T_0} + \ep \pdv[2]{u_1}{T_0} +
    2 \ep \pdv{u_0}{T_0}{T_1} + u_0 +
    \ep u_1 + \ep u_0^3 + \dots = 0
\end{equation*}

Приравнивая нулю соответствующие коэффициенты при $\ep^0$ и $\ep^1$,
имеем:
\begin{align}
    \pdv[2]{u_1}{T_0} + u_0 & = 0, \label{Eq:ep0} \\ 
    \pdv[2]{u_1}{T_0} + u_1 & = -2 \pdv{u_0}{T_0}{T_1} - u_0^3. \label{Eq:ep1} 
\end{align}

Общее решение уравнения \eqref{Eq:ep0}
может быть представлено в виде:
\begin{equation} \label{Eq:u0}
    u_0 = a(T_1, T_2, \dots) \cos\left[T_0 + \beta(T_1, T_2, \dots)\right]. 
\end{equation}

Отметим, что в данном случае $a$ и $\beta$ оказываются
не постоянными величинами, а функциями медленных масштабов $T_1, T_2, \dots$,
поскольку $u_0$ представляет собой функцию $T_0, T_1, T_2, \dots$,
а её производные в \eqref{Eq:ep0} берутся по переменной $T_0$.
На данной ступени аппроксимации функциональная зависимость
$a$ и $\beta$ от $T_1, T_2, \dots$ нам не известна и определяется
на последующих этапах путём исключения секулярных членов.

Подставляя теперь \eqref{Eq:u0} в \eqref{Eq:ep1}, получаем:
\begin{align}
    \pdv[2]{u_1}{T_0} + u_1 = &
    -2 \pdv{}{T_0}{T_1} \left[ a \cos(T_0 + \beta) \right]
    -a^3 \cos^3(T_0 + \beta) = \nonumber \\
    = & 2 \pdv{a}{T_1} \sin(T_0 + \beta) +
    2 a \pdv{\beta}{T_1} \cos(T_0 + \beta) - \nonumber \\
    - & \frac{3}{4} a^3 \cos(T_0 + \beta) -
    \frac{1}{4} a^3 \cos(3 T_0 + 3 \beta) = \nonumber \\
    = & 2 \pdv{a}{T_1} \sin(T_0 + \beta) +
    \left( 2 a \pdv{\beta}{T_1} - \frac{3}{4} a^3 \right)
    \cos(T_0 + \beta) - \frac{1}{4} a^3 \cos(3 T_0 + 3 \beta) \label{Eq:insert}
\end{align}

Неоднородность в правой части уравнения \eqref{Eq:insert}
служит источником секулярных членов для функции $u_1$.
В случае же равномерного разложения подобного рода члены
должны отсутствовать.
Этого можно добиться, приравняв нулю коэффициенты при
$\sin(T_0 + \beta)$ и $\cos(T_0 + \beta)$
в правой части \eqref{Eq:insert},
в результате чего получаем:
\begin{align}
    \pdv{a}{T_1} & = 0, \label{Eq:seq0} \\
    2 a \pdv{\beta}{T_1} - \frac{3}{4} a^3 & = 0. \label{Eq:seq1}
\end{align}

При этом частное решение уравнения \eqref{Eq:insert} принимает вид:
\begin{equation} \label{Eq:partAns}
    u_1 = \frac{1}{32} a^3 \cos(3 T_0 + 3 \beta). 
\end{equation}

Решение уравнения \eqref{Eq:seq0} есть функция
$a = a(T_2, T_3, \dots)$.
Тогда при $a \ne 0$ \eqref{Eq:seq1} можно переписать в виде уравнения:
\begin{equation*}
    \pdv{\beta}{T_1} = \frac{3}{8} a^2, 
\end{equation*}

решением которого будет функция:
\begin{equation} \label{Eq:beta}
    \beta = \frac{3}{8} a^2 T_1 + \beta_0(T_2, T_3, \dots). 
\end{equation}

Подставляя выражения для $u_0$ и $u_1$ из \eqref{Eq:u0}
и \eqref{Eq:partAns} в разложение \eqref{Eq:insertDecomp}, получаем:
\begin{equation} \label{Eq:decompU}
    u = a \cos(T_0 + \beta) + \frac{1}{32} \ep a^3 \cos(3 T_0 + 3 \beta) + \dots. 
\end{equation}

Подставляя теперь выражение для $\beta$ из \eqref{Eq:beta} в \eqref{Eq:decompU}
и вспоминая, что $a = a(T_2, T_3, \dots)$, находим:
\begin{align}
    u = & a(T_2, T_3, \dots) cos\left[ T_0 +
    \frac{3}{8} T_1 a^2(T_2, T_3, \dots)
    \beta_0(T_2, T_3, \dots) \right] + \nonumber \\
    + & \frac{1}{32} \ep a^3(T_2, T_3, \dots) \cos\left[ 3 T_0 + 
    \frac{9}{8} T_1 a^2(T_2, T_3, \dots) +
    3 \beta_0(T_2, T_3, \dots) \right] + \dots.  \label{Eq:uaT2}
\end{align}

Если в разложении \eqref{Eq:uaT2} ограничиться только выписанными членами,
то в рамках принятой точности функции $a$ и $\beta_0$ можно считать постоянными:
\begin{equation*}
    \begin{gathered}
        a(T_2, T_3, \dots) = a(\ep^2 t, \ep^3 t, \dots) = \\
        = a(0, 0, \dots) + \pdv{a}{T_2} \ep^2 t + \dots = \hat{a}_0 + O(\ep^2 t), \\
        \beta_0(T_2, T_3, \dots) = \beta_0(\ep^2 t, \ep^3 t, \dots) = \\
        \beta_0(0, 0, \dots) + \pdv{\beta_0}{T_2} \ep^2 t + \dots =
        \hat{\beta}_0 + O(\ep^2 t)
    \end{gathered}
\end{equation*}

Поэтому, заменяя $a$ и $\beta_0$ в \eqref{Eq:uaT2} постоянными величинами
$\hat{a}$ и $\hat{\beta}_0$, мы имеем:
\begin{align*}
    u = & \hat{a} \cos(T_0 + \frac{3}{8} T_1 \hat{a}^2 + \hat{\beta}_0) + \\
    + & \frac{1}{32} \ep \hat{a}^3 \cos(3 T_0 + \frac{9}{8} T_1 \hat{a}^2 +
    3 \hat{\beta}_0 ) + O(\ep^2 t),
\end{align*}

и после возвращения к исходной переменной $t$:
\begin{align}
    u = & \hat{a} \cos(t + \frac{3}{8} \ep t \hat{a}^2 + \hat{\beta}_0) + \nonumber \\
    + & \frac{1}{32} \ep \hat{a}^3 \cos(3 t + \frac{9}{8} \ep t \hat{a}^2 +
    3 \hat{\beta}_0 ) + O(\ep^2 t), \label{Eq:finalT}
\end{align}

Анализ формулы \eqref{Eq:finalT} показывает,
что ошибка в ней будет иметь порядок $O(1)$ и,
следовательно, окажется порядка первого члена,
если $t = O(\ep^{-2})$. Таким образом,
для значений $t \geq O(\ep^{-2})$ разложение \eqref{Eq:finalT},
становится непригодным. Если же $t = O(\ep^{-1})$,
то ошибка будет иметь порядок $\ep$,
т.е. окажется порядка второго члена разложения, и,
следовательно, разложение, пригодное при $t = O(\ep^{-1})$,
должно включать в себя только первый член.
Итак, для всех моментов времени вплоть до времен порядка $\ep^{-1}$:
\begin{equation*}
    u = \hat{a} \cos(t + \frac{3}{8} \ep t \hat{a}^2 + \hat{\beta}) + O(\ep). 
\end{equation*}

Это означает, что для того, чтобы построить равномерное разложение
первого порядка, мы должны, не решая самого уравнения
для $u_1$, исключить из него члены, служащие источником секулярных слагаемых,
найдя тем самым только зависимость $u_0$ от $T_1$.

Точно так же при построении высших приближений полагаем:
\begin{equation*}
    u = \sum_{n=0}^{N-1} \ep^n u_n(T_0, T_1, \dots, T_N) + O(\ep^N), 
\end{equation*}

т.е. если мы ищем разложение N-го порядка,
то должны учитывать масштабы $T_0, T_1, \dots, T_N$,
но не включить при этом в рассмотрение член порядка $\ep^N$.


