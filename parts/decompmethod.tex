\chapter*{Метод многомасштабных разложений}
\addcontentsline{toc}{section}{Метод многомасштабных разложений}

Для многих физических задач характерно
наличие малой силы или возмущения,
действующих в течение длительного времени.
Метод многомасштабных разложений разработан
для систематического исследования
такого кумулятивного эффекта.
Цель подобных методов состоит
в построении разложений,
равномерно пригодных на больших интервалах времени.
\cite{coul1972}

Основой метода --- представление
искомой зависимости $u(t)$ как сложной:
\begin{equation*}
    u = u(T_0(t, \ep), T_1(t, \ep), \dots),
\end{equation*}

где масштабы $T_0, T_1, \dots$
имеют разный порядок при $\ep \to 0$.
Часто достаточно принять
$T_0 = t, \, T_1 = \ep t, \, T_2 = \ep^2 t, \, \dots$

Оператор дифференцирования при простейшем выборе масштабов
становится следующим:
\begin{equation*}
    \dv{}{t} = \pdv{}{T_0} + \ep \pdv{}{T_1} +
    \ep^2 \pdv{}{T_2} + \dots,
\end{equation*}

из обыкновенного дифференциального уравнения
получается уравнение в частных производных.

Рассмотрим для примера уравнение:
\begin{equation*}
    \ddot{u} + u + \ep u^3 = 0
\end{equation*}

Выпишем его равномерное разложение:
\begin{equation*}
    u = a \cos(t + \beta + \frac{3}{8} \ep t a^2) +
    \frac{1}{32} \ep a^3
    \cos(3 t + 3 \beta + \frac{9}{8} \ep t a^2) + \dots
\end{equation*}

Видно, что мы не можем разделить
функциональную зависимость $u$ от $t$ и $\ep$,
поскольку функция $u$ зависит не только от этих аргументов
по отдельности, но и от произведения $\ep t$.
Если посмотреть на следующие члены более высокого порядка,
то можно понять, что $u$ зависит и от комбинаций
$\ep t, \, \ep^2 t, \, \ep^3 t, \, \dots$.

Это означает, что:
\begin{equation*}
    u(t, \ep) = u(t, \ep t, \ep^2 t, \ep^3 t, \ep) 
\end{equation*}

