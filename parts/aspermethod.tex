\chapter*{Метод асимптотических возмущений}
\addcontentsline{toc}{section}{Метод асимптотических возмущений}

Рассмотрим пример. Математическая формулировка многих задач, в которых
присутствуют функции вида (
), может быть дана с помощью дифференциального
уравнения вида
(
)
(1)
с некоторым граничным условием, где x – независимая переменная,
Если задача имеет точное решение при некотором
, то для малых
– параметр.
можно искать решение, например, в виде:
(
)
где
( )
( )
не зависит от ,
( )
,
(2)
( ) – решение задачи (1) при
. Разложение (2) при
подстановке в уравнение (1) и граничные условия должно удовлетворять им для всех
значений .
