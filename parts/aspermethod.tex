\chapter*{Метод асимптотических возмущений}
\addcontentsline{toc}{section}{Метод асимптотических возмущений}

Асимптотические методы возмущений
также называют методами малого параметра.
Рассмотрим сначала представление функций
через степенной ряд, затем обобщим на функции.

\section*{Возмущения по параметру}
\addcontentsline{toc}{subsection}{Возмущения по параметру}
Рассмотрим асимптотический метод возмущения по параметру.

Задачи, в которых задана функция $u(x, \varepsilon)$
могут быть сформулированы с помощью
дифференциального уравнения:
\begin{equation}
    \label{Eq:L}
    L(u, x, \varepsilon) = 0
\end{equation}
с ограничениями в виде:
\begin{equation}
    \label{Eq:B}
    B(u, \varepsilon) = 0,
\end{equation}
где $x$ --- скаляр или вектор, обозначающий независимую переменную,\\
а $\varepsilon$ - параметр задачи.

Если $\exists \, \varepsilon = \varepsilon_0$
такое, что задача решается, то для малых $\varepsilon$
можно получить решение разложением по степеням $\varepsilon$:
\begin{equation*}
   u(x, \varepsilon) = u_0(x) + \varepsilon u_1(x) + \
   \varepsilon^2 u_2(x) + \dots,
\end{equation*}
где $u_n$ не зависит от $\varepsilon$,\\
а $u_0(x)$ - решение задачи при $\varepsilon = 0$.

Это разложение затем можно подставить
в \eqref{Eq:L} и \eqref{Eq:B},
разложить аналогично по степеням $\varepsilon$,
и так как коэффициенты степеней обращаются в 0 независимо,
можно решить систему уравнений.

\clearpage
\section*{Возмущения по координате}
\addcontentsline{toc}{subsection}{Возмущения по координате}

Рассмотрим асимптотический метод возмущения по координате.\\
Пусть задача описывается дифференциальным уравнением:
\begin{equation*}
    L(u, x) = 0
\end{equation*}
с ограничениями:
\begin{equation*}
    B(u) = 0,
\end{equation*}
где $x$ --- скаляр.

Пусть известен вид $u_0$ решения $u$
при $x \to x_0$ (например, $x_0 = 0$).
Тогда можно попытаться найти отклонение $u$ от $u_0$
для $x$ близких к $x_0$,
раскладывая по степеням $x$ при $x = 0$.

\section*{Асимптотические разложения}
\addcontentsline{toc}{subsection}{Асимптотические разложения}

Как говорилось в начале раздела,
не обязательно раскладывать функцию по степенному ряду.

Можно использовать \textbf{асимптотическую последовательность} функции
$\delta_n(\varepsilon)$ с условием:
\begin{equation*}
    \delta_n(\varepsilon) = o[\delta_{n-1}(\varepsilon)] \
    \text{при}\ \varepsilon \to 0.
\end{equation*}
% TODO добавить про O-o асимптотику.

Как видно из определения,
степенной ряд является частным случаем
асимптотической последовательности функций.
Другими примерами асимптотических последовательностей могут являться:
\begin{equation*}
    \sin^n(\varepsilon), \quad \ctg^{-n}(\varepsilon), \quad
    \log^{-n}(\varepsilon), \quad \varepsilon^{n/3}
\end{equation*}

Назовём \textbf{асимптотическим разложением} сумму:
\begin{equation*}
    y \sim \sum_{m=0}^\infty a_m \delta_m(\varepsilon)
    \text{ при } \varepsilon \to 0,
\end{equation*}

тогда и только тогда, когда:
\begin{equation*}
    y = \sum_{m=0}^{n-1} a_m \delta_m(\varepsilon)
    + O[\delta_n(\varepsilon)]
    \text{ при } \varepsilon \to 0,
\end{equation*}

где $a_m$ не зависит от $\varepsilon$
и $\delta_m(\varepsilon)$ --- асимптотическая последовательность.

